\documentclass[compress,red]{beamer}
\mode<presentation>

%\usetheme[compress]{Boadilla}
\usetheme{Warsaw}
\setbeamerfont*{frametitle}{size=\normalsize,series=\bfseries}
\usefonttheme{serif}
\useoutertheme[subsection=false]{smoothbars}
\setbeamertemplate{navigation symbols}{}
\hypersetup{pdfpagemode=FullScreen} % automatic full screen

\usepackage{subfigure}
\usepackage[latin1]{inputenc}
\usepackage[english]{babel}
\usepackage{times}
\usepackage[T1]{fontenc}
\usepackage{multicol}
\usepackage{amsmath}
\usepackage{epsfig}
\usepackage{graphicx}
\usepackage{url}
\usepackage{multimedia}
\usepackage{hyperref}

\definecolor{amit}{rgb}{0,1,.5}
\definecolor{Red}{rgb}{1,0,0}
\definecolor{Blue}{rgb}{0,0,1}
\definecolor{Green}{rgb}{0,1,0}
\definecolor{magenta}{rgb}{1,0,.6}
\definecolor{lightblue}{rgb}{0,.5,1}
\definecolor{lightpurple}{rgb}{.6,.4,1}
\definecolor{gold}{rgb}{.6,.5,0}
\definecolor{orange}{rgb}{1,0.4,0}
\definecolor{hotpink}{rgb}{1,0,0.5}
\definecolor{newcolor2}{rgb}{.5,.3,.5}
\definecolor{newcolor}{rgb}{0,.3,1}
\definecolor{newcolor3}{rgb}{1,0,.35}
\definecolor{darkgreen1}{rgb}{0, .35, 0}
\definecolor{darkgreen}{rgb}{0, .6, 0}
\definecolor{darkred}{rgb}{.75,0,0}
\xdefinecolor{olive}{cmyk}{0.64,0,0.95,0.4}
\xdefinecolor{purpleish}{cmyk}{0.75,0.75,0,0}


\def\pa{\partial}
\newcommand{\ta}{\tau}
\newcommand{\na}{\bf \nabla}
\newcommand{\be}{\begin{equation*}}
\newcommand{\ee}{\end{equation*}}
\newcommand{\bea}{\begin{eqnarray*}}
\newcommand{\eea}{\end{eqnarray*}}

\newcommand{\stl}[1]{\mbox{$ \hspace{0.1em}
      \stackrel{\rule{0.4pt}{0.275ex}\hspace{0.40em} \!\!\!
      \overline{\hspace{0.06em}\vphantom{\rule{0.4pt}{0.0ex}}
      \hphantom{\mbox{$\displaystyle #1$}}
      \hspace{0.06em} } \!\!\!\hspace{0.40em}\rule{0.4pt}{0.275ex}}
      {#1}\hspace{0.2em}$}}

\title[\tiny Expert Committee Interview]{Some Outstanding Problems in Liquid Crystal Physics}
\author{Amit Kumar Bhattacharjee}
\institute{Asutosh College, Kolkata} 
\date{\today}

\begin{document}
\begin{frame}
\titlepage
\end{frame}

\section{Outline}

\frame{\frametitle{Organization}  
\vspace{-.5cm}
\begin{enumerate}
\item Background of nematic mesophases; statics and kinetics. 
\item Numerical techniques and benchmarks.
\item Biaxiality of the isotropic-nematic interface; effect of rotational anchoring.
\item Shape of nematic bubble in isotropic background.
\item Phase ordering through spinodal kinetics.
\item Ongoing work.
\item Publications.
\end{enumerate}
}

\section{Introduction}
 
\frame{\frametitle{Background of Nematogens}
\begin{itemize}      
\item Anisotropic molecules (rods,discs) having long range orientational order devoid of translational order.
\item Rotational symmetry about the direction of order, {\it uniaxial} phase 
(${\bf n} \leftrightarrow -{\bf n}$).
\item No rotational symmetry : {\it biaxial} order(${\bf n} \leftrightarrow -{\bf n}$,
${\bf l} \leftrightarrow -{\bf l}$).
\item Alignment tensor order have five degrees of freedom, 2 degrees of order and 3 angles to specify 
      principal direction. 
\item $Q_{ij} = \frac{3}{2}S(n_in_j - \frac{1}{3}\delta_{ij}) + \frac{T}{2}(l_il_j - m_im_j) (i,j=x,y,z)$. 
\end{itemize}

\begin{figure}
\centering
\includegraphics[width=1.4in, height=1.0in, angle=0]{/home/amit/Research/latex/docinterview/24.07.09/lc.pdf} 
\hspace{0.3 in}
\includegraphics[width=1.4in, height=1.0in, angle=0]{/home/amit/Research/latex/docinterview/24.07.09/biaxial.pdf} \\
\end{figure}
}

\frame{\frametitle{Statics : Free energy, phase diagram}
\begin{eqnarray*}
\mathcal{F}[{\bf Q},{\pmb\nabla \bf Q}] &=& \int d^3{\bf x} [\frac{1}{2}ATr{\bf Q}^{2} + \frac{1}{3}BTr{\bf Q}^{3} + 
\frac{1}{4}C(Tr{\bf Q}^{2})^{2} + E^{\prime}(Tr{\bf Q}^{3})^{2} + \nonumber \\ 
&& \frac{1}{2}L_{1}(\partial_{\alpha}Q_{\beta\gamma})(\partial_{\alpha}Q_{\beta\gamma}) 
+ \frac{1}{2}L_{2}(\partial_{\alpha}
Q_{\alpha\beta})(\partial_{\gamma}Q_{\beta\gamma})].
\end{eqnarray*}

\begin{figure}
\includegraphics[width=2.1in, height=1.6in, angle=0]{/home/amit/Research/latex/docinterview/24.07.09/frener.pdf}
\includegraphics[width=2.1in, height=1.8in]{/home/amit/Research/latex/docinterview/24.07.09/phasediagram.pdf}
\end{figure}
}

\frame{\frametitle{Kinetics}
\begin{itemize}
\item Landau-Ginzburg model-A dynamics for non-conserved order parameter. \\
\item $\partial_{t}Q_{\alpha\beta}({\bf x}, t) = - \Gamma_{\alpha\beta\mu\nu} {\delta \mathcal{F}\over\delta Q_{\mu\nu}}$, ~~\texttt{where}\\
      $\Gamma_{\alpha\beta\mu\nu} = \Gamma[\delta_{\alpha\mu}\delta_{\beta\nu} + \delta_{\alpha\nu}\delta_{\beta\mu} - \frac{2}{d}\delta_{\alpha\beta}\delta_{\mu\nu}]$. \\ \ \\ 

\small
\begin{block}{}
$\partial_{t}Q_{\alpha\beta}({\bf x}, t) = - \Gamma \;[(A + C TrQ^{2})Q_{\alpha\beta}({\bf x}, t) +$ \\ 
$(B + 6E^{\prime}TrQ^{3})\stl{Q^{2}_{\alpha\beta}({\bf x}, t)} - L_{1}\nabla^{2}Q_{\alpha\beta}({\bf x}, t) - 
L_{2}\stl{\nabla_{\alpha}(\nabla_{\gamma}Q_{\beta\gamma}({\bf x}, t))}]$ \\  
\end{block}

\item Route to equilibrium $\Rightarrow$ \textcolor{blue}{nucleation} kinetics above $T^{*}$, \textcolor{blue}{spinodal} 
      kinetics beneath $T^{*}$. 
\item $Q_{\alpha\beta}({\bf x}, t) = \sum_{i=1}^{5}a_{i}({\bf x}, t)T^{i}_{\alpha\beta}$, 
      \small ${\bf T}^{1} = \sqrt{\frac{3}{2}} \stl{\bf z \; z}, {\bf T}^{2} =  \sqrt{\frac{1}{2}} ({\bf x \; x - y \; y}),
       {\bf T}^{3} = \sqrt{2}\; \stl{{\bf x \; y}}, {\bf T}^{4} = \sqrt{2}\; \stl{{\bf x \; z}},$ 
       ${\bf T}^{5} = \sqrt{2}\; \stl{{\bf y \; z}}$.
\end{itemize}
}

\frame{\frametitle{Problems at a glance}
\begin{figure}
\includegraphics[width=3.5in, height=2.5in, angle=0]{/home/amit/Research/latex/docinterview/24.07.09/flowchart.pdf}
\end{figure}
}

\section{Numerics}

\frame{\frametitle{Numerical techniques}
\begin{itemize}
\item Method of lines 
      \begin{itemize}
      \item Spatial finite difference discretization.
      \item Temporal integration using standard library.
      \item Benchmark of {\it tanh} interface, ellipsoidal droplet, corsening. 
      \item Performed in 2D on lattices, ranging from $256^2$ to $1024^2$.
      \item Performed in 3D on lattices, ranging from $64^3$ to $256^3$.
      \end{itemize}
\item Spectral methods
      \begin{itemize}
      \item Space discretized on chebyshev grids $x_j=cos(\pi j/N)$.
      \item Global interpolation retaining the spectral accuracy.
      \end{itemize}
\item High-performance computation
      \begin{itemize}
      \item Domain decomposition of the differentiation matrix and vector on a parallel 
            cluster using standard library. 
      \item Structured binary data storage using standard library.
      \end{itemize}
\end{itemize}
}

\section{IN interface}

\frame{\frametitle{Isotropic-Nematic interface}
\begin{itemize}
\item Verification of ``de Gennes ansatz'' and limitations using method of lines.
\item Biaxial nature of IN interface with planar anchoring using spectral method.
\begin{figure}
\subfigure[$\; \kappa = 0 $]{
    \centering
    \includegraphics[width=1.8in, height=1.4in, angle=0]{/home/amit/Research/latex/docinterview/24.07.09/fig2_interface.pdf}
}
\subfigure[$\; \kappa = 18 $]{
    \centering
    \includegraphics[width=1.8in, height=1.4in, angle=0]{/home/amit/Research/latex/docinterview/24.07.09/AKSTnew18.pdf}
}
\end{figure}
\end{itemize} 
}

\frame{\frametitle{Contd..}
\begin{itemize} 
\item Director anchoring at the interface with tilted anchoring at boundary.
\end{itemize} 
\begin{figure}
\subfigure[$\; \kappa = 36 $]{
     \centering
     \includegraphics[width=1.8in, height=1.4in, angle=0]{/home/amit/Research/latex/docinterview/24.07.09/anc1.pdf}
}
\hspace{0.35 cm}
\subfigure[$\; \kappa = 36 $]{
     \centering
     \includegraphics[width=1.8in, height=1.4in, angle=0]{/home/amit/Research/latex/docinterview/24.07.09/anc3.pdf}
}
\end{figure}
}

\section{Droplet}

\frame{\frametitle{Nematic droplet in isotropic background}
\begin{itemize}
\item Nematic bubble grow or shrink in the nucleation regime. 
\item Contribution from the anisotropic surface tension $\Rightarrow$ shape change from circular to ellipsoidal.
\item No approximation of surface free energy which automatically included in our formulation.
\item Consequences : nucleation rate ($\propto e^{-B/k_BT}$) can be calculated exactly, apart from the prefactors.

\begin{figure}
\subfigure[$L_2 = 0 $]{
     \centering
     \includegraphics[width=0.8in, height=0.8in, angle=0]{/home/amit/Research/latex/docinterview/24.07.09/circular.pdf}
}
\hspace{0.35 cm}
\subfigure[$L_2 = 10L_1$]{
     \centering
     \includegraphics[width=0.8in, height=0.8in, angle=0]{/home/amit/Research/latex/docinterview/24.07.09/ellip1.pdf}
}
\hspace{0.35 cm}
\subfigure[$L_2 = -L_1$]{
     \centering
     \includegraphics[width=0.8in, height=0.8in, angle=0]{/home/amit/Research/latex/docinterview/24.07.09/homo.pdf}
}
\end{figure}
\end{itemize} 
}

\section{Corsening}

\frame{\frametitle{Phase ordering kinetics}
\setbeamercovered{transparent}
\onslide<1->
\underline{2D} \\
\onslide<2->
\begin{enumerate}
\item Visualization and topological classification of point defects. 
\item Structure of defect core of different homotopy class.
\item Dynamical scaling exponent. \\
\end{enumerate} 

\onslide<3->
\underline{3D} \\
\begin{enumerate}
\item Line defects in nematics; intercommutation of defect segments.
\item Director configuration around the segment.
\item Topological rigidity in biaxial nematics.
\end{enumerate} 
}

\frame{\frametitle{Defects in nematics}
\begin{itemize} 
\item \textcolor{blue}{uniaxial} nematic defects are characterized through 
$\pi_1(\mathbb{S}^2/\mathbb{Z}_2) = \mathbb{Z}_2$, having unstable integer and 
stable half integer charged defects.
\item \textcolor{blue}{biaxial} nematic defects are characterized through 
$\pi_1(\mathbb{S}^3/\mathbb{D}_2) = \mathbb{Q}_8$, having a stable integer ($\bar C_0$ 
class, $2\pi$ rotation of director) and three half-integer ($C_x, C_y, C_z$, $\pi$ rotation 
of director) charged defects.
\item Defects are visualized and classified through scalar order (movie).
\item Textures (intensity $\propto sin^2[2\theta]$) show a subset while all the half-integer defect locations are identified in $S({\bf x}, t), T({\bf x},t)$.

\begin{figure}
\subfigure[]{
     \centering
     \includegraphics[width=0.6in, height=0.6in, angle=0]{/home/amit/Research/latex/docinterview/24.07.09/Suniax.pdf}
}
\hspace{0.35 cm}
\subfigure[]{
     \centering
     \includegraphics[width=0.62in, height=0.62in, angle=0]{/home/amit/Research/latex/docinterview/24.07.09/2dS_00000100.pdf}
}
\hspace{0.35 cm}
\subfigure[]{
     \centering
     \includegraphics[width=0.62in, height=0.62in, angle=0]{/home/amit/Research/latex/docinterview/24.07.09/2dSch_00000100.pdf}
}
\end{figure}
\end{itemize} 
}

\frame{\frametitle{Core structure; dynamical scaling}
\begin{figure}
\subfigure[]{
     \centering
     \includegraphics[width=1.3in, height=1.1in, angle=0]{/home/amit/Research/latex/docinterview/24.07.09/fig5c_defprof.pdf}
}
\subfigure[$C_x$]{
     \centering
     \includegraphics[width=1.3in, height=1.1in, angle=0]{/home/amit/Research/latex/docinterview/24.07.09/ST1.pdf}
}
\subfigure[$C_y$]{
     \centering
     \includegraphics[width=1.3in, height=1.1in, angle=0]{/home/amit/Research/latex/docinterview/24.07.09/ST2.pdf}
}
\end{figure}
\begin{itemize}
\item Uniaxial dynamical scaling exponent $\alpha = 0.5\pm0.005 \; [L(t)\sim t^{\alpha}]$.
\end{itemize}
}

\frame{\frametitle{Line defects in 3D}
\begin{itemize}
\item Point defects in 2D correspond to strings in 3D.
\item Annihilation of point defect-antidefect correspond to formation and disappearance of loop.
\item Line defects pass through each other through intercommutation i.e. exchanging segments (movie ; isosurface set to 0.054).
\item Intercommutation of lines depend on the underlying abelian nature of the group elements of that 
      particular homotopy group (Poenaru et.al. '77).
\item No such signature seen in biaxial nematics !! 

\begin{figure}
\subfigure[]{
     \centering
     \includegraphics[width=1.0in, height=1.0in, angle=0]{/home/amit/Research/latex/docinterview/24.07.09/3dSN_biax_45.pdf}
}
\subfigure[]{
     \centering
     \includegraphics[width=1.0in, height=1.0in, angle=0]{/home/amit/Research/latex/docinterview/24.07.09/3dSN_biax_49.pdf}
}
\subfigure[]{
     \centering
     \includegraphics[width=1.0in, height=1.0in, angle=0]{/home/amit/Research/latex/docinterview/24.07.09/3dSN_biax_51.pdf}
}
\end{figure}
\end{itemize}
}

\section{Future work}

\frame{\frametitle{Ongoing work}
\begin{itemize}
\item Nucleation kinetics in fluctuating nematics; nematic bubbles in 3D.
\item Scaling exponent in 3D uniaxial and biaxial coarsening nematic (d=3,n=3).
\item Scaling exponent of uniaxial nematic with space and spin dimension 2 (d=2,n=2).   
\item Topological rigidity in biaxial nematics ? Interplay of energetics over topology. 
\end{itemize}
}

\frame{\frametitle{Publications}
\begin{itemize}
\item Method of lines for the relaxational dynamics of nematic liquid crystals, PRE {\bf 78}, 026707 (2008).
\item Biaxiality at the isotropic-nematic interface with planar anchoring, arXiv : 0906.2899 (submitted to PRE, 
Rapid Comm.).
\item Simulation and visualization of disclinations in nematic liquid crystals (to be submitted in ``Soft Matter'').
\item Nucleation kinetics in fluctuating Landau-de Gennes theory for uniaxial nematics (in preparation).
\item Effect of general anchoring of the director on the isotropic-nematic interface (in preparation).
\end{itemize}
}

\begin{frame}
\begin{center}
{\bf \large \textcolor{gray}{Thanks for your attention}} 
\end{center}
\end{frame}

\end{document}
